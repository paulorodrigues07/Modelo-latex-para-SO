\subsection{Criando processos}

Em suma os SO precisam de uma maneira de criar os processos. Podemos dizer de forma abstrata que o programa é como um livro de receita e que o processo é o ato de cozinhar em si onde se segue passo a passo da receita e que os ingredientes para o preparo são dos dados \cite{Tanenbaum2016}, \cite{info2020}, \cite{Morimoto2011}.\\
Em sistemas simples ou embarcados pode ser possível ter todos os processos que serão em algum momento necessários quando o sistema for ligado \cite{Tanenbaum2016}, \cite{info2020}, \cite{Morimoto2011}.\\
Mas para sistemas para fins gerais, no entanto, de alguma maneira é necessária para criar e terminar processos, na medida do necessário, durante a operação \cite{Tanenbaum2016}, \cite{info2020}, \cite{Morimoto2011}.\\

\begin{itemize}[label=$-$]
	\item Quatro eventos principais fazem com que os processos sejam criados:
    \begin{enumerate}
        \item Inicialização do sistema.
        \item Execução de uma chamada de sistema de criação de processo por um processo em execução.
        \item Solicitação de um usuário para criar um novo processo.
        \item Início de uma tarefa em lote. 
    \end{enumerate}
\end{itemize}

Durante o carregamento inicial do SO diversos processos são criados alguns em primeiro plano e outros em segundo plano. Podemos dizer que todos os processos de primeiro plano são de interação com o usuário e que os processo de segundo plano são de domínio do próprio sistema \cite{Tanenbaum2016}, \cite{info2020}, \cite{Morimoto2011}, \cite{Man2020}.\\
No linux server cada processo possui um número de PID (Process Identifier - em portugues Identificador de Processo)este é um número de identificação que o sistema dá a cada processo. Para cada novo processo, um novo número deve ser atribuído, ou seja, não se pode ter um único PID para dois ou mais processos ao mesmo tempo \cite{Tanenbaum2016}, \cite{info2020}, \cite{Morimoto2011}, \cite{Man2020}.\\
Os sistemas baseados em Unix precisam que um processo já existente se duplique para que a cópia possa ser atribuída a uma tarefa nova. Quando isso ocorre, o processo "copiado" recebe o nome de "processo pai", enquanto que o novo é denominado "processo filho". É nesse ponto que o PPID (Parent Process Identifier - em portugues Identificador de processo pai ) passa a ser usado: o PPID de um processo nada mais é do que o PID de seu processo pai \cite{Tanenbaum2016}, \cite{info2020}, \cite{Morimoto2011}, \cite{Man2020}.\\
No UNIX, há apenas uma chamada de sistema para criar um novo processo: fork. Essa chamada cria um clone exato do processo que a chamou. Após a fork, os dois processos, o pai e o filho, têm a mesma imagem de memória, as mesmas variáveis de ambiente e os mesmos arquivos abertos. E isso é tudo. Normalmente, o processo filho então executa execve ou uma chamada de sistema similar para mudar sua imagem de memória e executar um novo programa  \cite{Tanenbaum2016}, \cite{info2020}, \cite{Morimoto2011}, \cite{Man2020}.