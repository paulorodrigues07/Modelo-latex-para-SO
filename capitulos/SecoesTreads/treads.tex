\newpage
\section{Threads}\label{sec:Threads}


Cada programa, ou processo, possui normalmente um fluxo de controle. Assim o programa é executado seqüencialmente passo a passo com seu único fluxo de controle. Em sistemas operacionais tradicionais, cada processo tem um espaço de endereçamento e um único \emph{thread} de controle. Neste ponto que as \emph{threads} se destacam, com as \emph{threads} podemos ter mais de um único fluxo de controle em nosso aplicativo \cite{Tanenbaum2016}, \cite{dev2020}.\\
As \emph{threads} em  muitas situações, é desejável ter múltiplos \emph{threads} de controle no mesmo espaço de endereçamento executando em quase paralelo, como se eles fossem (quase) processos separados (exceto pelo espaço de endereçamento compartilhado) \cite{Tanenbaum2016}, \cite{dev2020}.\\
Assim o \emph{software} agira como se tivessem sido dividido em varias partes de seu código atuando em paralelo no sistema. \emph{Threads} são, portanto, entidades escalonadas para executarem na \emph{CPU}, por isso a noção de (pseudo) paralelismo, pois as \emph{threads} concorrerão pelo processador juntamente com mais \emph{threads} que tiverem no programa, ou concorrerá apenas com o fluxo do programa \cite{Tanenbaum2016}, \cite{dev2020}.
