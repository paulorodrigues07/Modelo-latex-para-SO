\newpage
\section{Threads}\label{sec:Threads}


Cada programa, ou processo, possui normalmente um fluxo de controle. Assim o programa é executado seqüencialmente passo a passo com seu único fluxo de controle. Em sistemas operacionais tradicionais, cada processo tem um espaço de endereçamento e um único thread de controle. Neste ponto que as threads se destacam, com as threads podemos ter mais de um único fluxo de controle em nosso aplicativo \cite{Tanenbaum2016}, \cite{dev2020}.\\
As threads em  muitas situações, é desejável ter múltiplos threads de controle no mesmo espaço de endereçamento executando em quase paralelo, como se eles fossem (quase) processos separados (exceto pelo espaço de endereçamento compartilhado) \cite{Tanenbaum2016}, \cite{dev2020}.\\
Assim o software agira como se tivessem sido dividido em varias partes de seu código atuando em paralelo no sistema. Threads são, portanto, entidades escalonadas para executarem na CPU, por isso a noção de (pseudo) paralelismo, pois as threads concorrerão pelo processador juntamente com mais threads que tiverem no programa, ou concorrerá apenas com o fluxo do programa \cite{Tanenbaum2016}, \cite{dev2020}.
