\section{O zoológico dos sistemas operacionais}
%\addcontentsline{toc}{section}{Estrutura de hardware
Os sistemas operacionais e existem a mais de meio século e durante esse tempo diversos sistemas foram criados. E esses sistemas foram criados para atender a distintas finalidades.\\
Existem diversas formas e abordagens para a separação destes sistemas operacionais exemplificaremos apenas algumas mais conhecidas.

\subsection{Sistemas operacionais de computadores de uso pessoal}

Os SO destinados a essa função dão suporte a multiprocessos e são multiusuários normalmente durante seu início do sistema ele carrega diversos programas que proporcionem um apoio ao usuário. é comum que os SO de computadores pessoais possua uma interface gráfica (GUI) pois o foco deste sistema é proporcionar um bom apoio ao usuário \cite{Tanenbaum2016}, \cite{Comer2012}.\\

\subsection{Sistemas operacionais de servidores}

Assim como os sistemas operacionais de computadores de uso pessoal esses sistemas são multiusuário e executadas multi processos mas diferentemente do sistema anterior normalmente ele não possui um GUI e sim apenas um shell de acesso \cite{Tanenbaum2016}, \cite{Comer2012}.\\
Eles são executados em servidores que são computadores pessoais muito grandes, em estações de trabalho ou mesmo computadores de grande porte \cite{Tanenbaum2016}, \cite{Comer2012}.\\
Servidores são utilizados principalmente para fornecer serviços como  de impressão, de arquivo ou de \emph{web}. Provedores de acesso à internet utilizam várias máquinas servidoras para dar suporte aos clientes, e sites usam servidores para armazenar páginas e lidar com as requisições que chegam \cite{Tanenbaum2016}, \cite{Comer2012}.\\
O sistema que trataremos neste trabalho é um sistema de servidor \cite{Tanenbaum2016}, \cite{Comer2012}.\\

\subsection{Sistemas operacionais embarcados}

Sistemas embarcados são executados em computadores que controlam dispositivos que não costumam ser vistos como computadores e que não aceitam \emph{softwares} instalados pelo usuário. Exemplos típicos são os fornos de micro-ondas, os aparelhos de televisão, os carros, os aparelhos de \emph{DVD}, os telefones tradicionais e os \emph{MP3 players}. A principal propriedade que distingue sistemas embarcados dos portáteis é a certeza de que nenhum \emph{software} não confiável vá ser executado nele um dia. Você não consegue baixar novos aplicativos para o seu forno de micro-ondas – todo o \emph{software} está na memória \emph{ROM}. Isso significa que não há necessidade para proteção entre os aplicativos, levando a simplificações no \emph{design} \cite{Tanenbaum2016}, \cite{Comer2012}.\\ 
