\subsection{Barramentos}

A placa mãe é conhecida por comportar todos os periféricos do computador seja as memórias ou a CPU mais sua principal funçao é comportar os barramentos. O barramento é a linha de comunicação entre esses dispositivos. Normalmente os barramentos são divididos por funções específicas ou comunicações específicas como barramento de memória ou de dispositivos de E/S \cite{Tanenbaum2016}, \cite{Comer2012}.\\
Barramento de dados – como o próprio nome já deixa a entender, é por este tipo de barramento que ocorre as trocas de dados no computador, tanto enviados quanto recebidos \cite{Tanenbaum2016}, \cite{Comer2012}.\\
Barramento de endereços – indica o local onde os processos devem ser extraídos e para onde devem ser enviados após o processamento \cite{Tanenbaum2016}, \cite{Comer2012}.\\
Barramento de controle – atua como um regulador das outras funções, podendo limitá-las ou expandi-las em razão de sua demanda \cite{Tanenbaum2016}, \cite{Comer2012}.\\
Barramentos de entrada e saída – atua fazendo a ligação com dispositivos de E/S \cite{Tanenbaum2016}, \cite{Comer2012}.\\
Barramento de memória – atua diretamente na troca de informação da memória e um dos aspectos fundamentais quanto a esse  barramento e a se tratando da quantidade de bits que ele é capaz de levar por vez. Um ótimo exemplo para elucidar isso é em relação à placas de vídeo \cite{Tanenbaum2016}, \cite{Comer2012}.\\
