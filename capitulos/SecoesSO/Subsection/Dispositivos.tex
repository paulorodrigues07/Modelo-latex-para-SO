\subsection{Dispositivos de E/S}
O sistema operacional não trabalha apenas com CPU e memórias ele tem que gerenciar outros tipos de hardware estes equipamentos em questão tem funções específicas mas basicamente ele trabalham fazendo o \emph{input} e \emph{output} de informações no sistema e são conhecidos como dispositivos de E/S. podemos dizer que estes dispositivos possuem duas características básicas um controlador e o dispositivo em si \cite{Tanenbaum2016}, \cite{Comer2012}.\\
O controlador ele proporciona a comunicação entre o SO e o dispositivo em si, como cada dispositivo possui características particulares que podem ser influenciadas pelo fabricante ou tecnologia empregada assim se faz necessário que seja disponibilizado para o sistema \emph{software} auxiliares que ajudam o SO a comunicar com o controlador esse \emph{software} são chamados de Drivers \cite{Tanenbaum2016}, \cite{Comer2012}.\\
O dispositivo em si normalmente possui padronização de saídas físicas para que as conexões com o computador sejam simplificadas e por exemplo que uma saída sata seja idêntica independente do fabricante ou da tecnologia empregada para a construção do dispositivo \cite{Tanenbaum2016}, \cite{Comer2012}.