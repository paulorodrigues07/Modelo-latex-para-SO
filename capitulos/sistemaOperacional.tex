\chapter[Sistema Operaciona]{Sistema Operacional}
%\addcontentsline{toc}{chapter}{Sistema Operacional}

Mas afinal o que é um sistema operacional?

É difícil entender o que é um sistema operacional pois talvez a única coisa que podemos afirmar é que é um \emph{software} que trabalhe em modo núcleo (e por vezes até isso não é uma completa verdade), para tal precisamos entender que o SO tem duas funções bases que é fornecer uma abstração do \emph{hardware} para programadores de aplicativos e usuários em geral, e o gerenciamento dos recursos de \emph{hardware}.\\
 Em geral o conjunto de instruções, estrutura de barramento, E/S e a organização de memória é complexo e varia dependendo da arquitetura das peças utilizadas no computador. Por esse motivo o SO fornece uma camada de abstração para os aplicativos que se utilizam dos \emph{hardwares} do computador possam criar, escrever e ler arquivos, sem ter de lidar com os detalhes complexos de como o \emph{hardware} realmente funciona.\\
Como dito anteriormente um computador moderno consiste em um emaranhado de peça e a função do sistema operacional é fornecer uma alocação ordenada e controlada para todas elas além que o SO moderno permite que múltiplas aplicações estejam em memória e sejam executados “ao mesmo tempo”. \\
Precisamos entender que o sistema não faz  execução de todos os recursos ao mesmo tempo isso seria insano de se imaginar mais ele utiliza de filas de processos e de um nivelamento de urgências para definir o que será processado e quando será processado, ele ainda define as utilizações em nível de memória para definir prioridades e utilizando delas para alterar o tempo  de acesso ao processador, desta forma mesmo com vários processos “abertos” a execução se dá em filas multiplexadas.\\
Ainda devemos entender que se o sistema possuir vários usuários a necessidade de gerenciar e proteger a memória, dispositivos de E/S e outros recursos é ainda maior, pois cada usuário precisa ter acesso aos recursos de \emph{hardwares} e compartilhar de informações salvas como (arquivos, bancos de dados etc.) e as ações de um usuário pode influenciar ao uso do outro. \\
Podemos dizer então que a função primordial do SO é manter controle, isso seja sobre como conceder acesso a recursos requisitados, sobre quais programas estão usando qual recurso, sobre como conceder recursos requisitados, contabilizar o seu uso, assim como mediar requisições conflitantes de diferentes programas e usuários.\\
O termo \emph{hardware} foi muito utilizado mais qual seriam os \emph{hardwares} de um computador ou uma estrutura basica dos mesmos e como o sistema operacional se utiliza deles para em sua execução. 

