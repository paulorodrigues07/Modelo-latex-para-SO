% ---
% RESUMOS
% ---

% RESUMO em português
\setlength{\absparsep}{18pt} % ajusta o espaçamento dos parágrafos do resumo
\begin{resumo}
    O objetivo deste presente trabalho é a consolidação dos conhecimentos adquiridos durante a realização da disciplina de Sistemas Operacionais, dando enfoque aos sistemas baseados em Linux utilizados em servidores, salientando suas utilizações e o seu mercado de atuação. \\
    O Linux é um sistema operacional que vive em um crescimento contínuo e é amplamente usado ele está tanto em sensores a supercomputadores, e podemos vê-lo sendo usados em espaçonaves, automóveis, \emph{smartphones}, relógios e muitos outros dispositivos em nossa vida cotidiana \cite{LinuxFundationWIL}.\\
    Em especial o sistema Linux é um sistema de código aberto o que significa que é possível executá-lo para qualquer propósito, estudar seu funcionamento e modificá-lo se assim desejar, ou realizar cópias para terceiros dando total liberdade para sua comunidade \cite{LinuxFundationWIL},\cite{Morimoto2011}.\\
    Ele também opera a maior parte da Internet, todos os 500 maiores supercomputadores do mundo e as bolsas de valores do mundo. Estes funcionam em uma variação do Linux preparada para um grande fluxo de tratamento de dados, podendo rodar vários serviços simultaneamente, esta versão é a Linux para servidores ou \emph{Linux Server} \cite{LinuxFundationWIL},\cite{Morimoto2011}.\\





 \textbf{Palavras-chaves}: Linux, Servidores, Sistema Operacional.
\end{resumo}

% ABSTRACT in english
% \begin{resumo}[Abstract]
%  \begin{otherlanguage*}{english}
%   This is the english abstract.

%   \vspace{\onelineskip}
 
%   \noindent 
%   \textbf{Keywords}: latex. abntex. text editoration.
%  \end{otherlanguage*}
% \end{resumo}